\documentclass[10pt,twocolumn,letterpaper]{article}

\usepackage{cvpr}
\usepackage{times}
\usepackage{epsfig}
\usepackage{graphicx}
\usepackage{amsmath}
\usepackage{amssymb}
\DeclareUnicodeCharacter{FB01}{fi}
% Include other packages here, before hyperref.

% If you comment hyperref and then uncomment it, you should delete
% egpaper.aux before re-running latex.  (Or just hit 'q' on the first latex
% run, let it finish, and you should be clear).
\usepackage[breaklinks=true,bookmarks=false]{hyperref}
\usepackage{float}
\cvprfinalcopy % *** Uncomment this line for the final submission

\def\cvprPaperID{****} % *** Enter the CVPR Paper ID here
\def\httilde{\mbox{\tt\raisebox{-.5ex}{\symbol{126}}}}

% Pages are numbered in submission mode, and unnumbered in camera-ready
%\ifcvprfinal\pagestyle{empty}\fi
\setcounter{page}{4321}
\begin{document}

%%%%%%%%% TITLE
\title{\LaTeX\ Joint Training of Cascaded CNN for Face Detection}
\iffalse
\author{First Author\\
Institution1\\
Institution1 address\\
{\tt\small firstauthor@i1.org}
% For a paper whose authors are all at the same institution,
% omit the following lines up until the closing ``}''.
% Additional authors and addresses can be added with ``\and'',
% just like the second author.
% To save space, use either the email address or home page, not both
\and
Second Author\\
Institution2\\
First line of institution2 address\\
{\tt\small secondauthor@i2.org}
}
\fi

\author {Hongwei Qin   Junjie Yan   Xiu Li   Xiaolin Hu}


\maketitle
%\thispagestyle{empty}

%%%%%%%%% ABSTRACT
\begin{abstract}
   Cascade has been widely used in face detection, where
classifier with low computation cost can be firstly used to
shrink most of the background while keeping the recall. The
cascade in detection is popularized by seminal Viola-Jones
framework and then widely used in other pipelines, such
as DPM and CNN. However, to our best knowledge, most
of the previous detection methods use cascade in a greedy
manner, where previous stages in cascade are fixed when
training a new stage. So optimizations of different CNNs are
isolated. In this paper, we propose joint training to achieve
end-to-end optimization for CNN cascade. We show that
the back propagation algorithm used in training CNN can
be naturally used in training CNN cascade. We present how
jointly training can be conducted on naive CNN cascade
and more sophisticated region proposal network (RPN) and
fast R-CNN. Experiments on face detection benchmarks verify the advantages of the joint training
\end{abstract}

%%%%%%%%% BODY TEXT




\iffalse

\subsection{References}

List and number all bibliographical references in 9-point Times,
single-spaced, at the end of your paper. When referenced in the text,
enclose the citation number in square brackets, for
example~\cite{Authors14}.  Where appropriate, include the name(s) of
editors of referenced books.
\fi


\section{ExperimentsofjointlytrainedfasterR-CNN }

\makeatletter\def\@captype{table}\makeatother
%\begin{table}
\begin{center}
\begin{tabular}{ccc}
\hline
Benchmark & Separate & Joint\\
\hline
AFW & 97.0\% & 98.7\% \\
\hline
FDDB & 89.7\% & 91.2\% \\
\hline
\end{tabular}
\end{center}
\caption{ Comparison of training methods of RPN + F-RCNN }
%\end{table}
As shown in Table. 1, with our presented RPN + FRCNN (fast R-CNN) joint training pipeline, the AP (average precision) on AFW is 98.7\%, compared to the baseline result 97.0\% trainedwith4-stagetrainingmethodproposed in[21].
OnFDDB,therecall(1000falsepositives)is 91.2\% and 89.7\%. For the F-RCNN branch, the final joint training loss decreases 64\% compared to separate training. In joint RPN + F-RCNN, the detection results mostly have much higher confidence scores than separate training results, which have lower confidence scores because of FRCNN domination in convolution layers.


%------------------------------------------------------------------------
\section{Jointloss}
 Each branch has a face v.s. non-face classfication loss and a bounding-box regression loss. Adding them with loss weights, we get the joint loss function:

\begin{center}
$L_{point} = \lambda_{1}L_{x12}+ \lambda_{2}L_{x24} + \lambda_{3}L_{x48}$
\end{center}


%------------------------------------------------------------------------
\section{AFW results}
\begin{figure}[H]
\begin{center}
%\fbox{\rule{0pt}{2in} \rule{0.9\linewidth}{0pt}}
   \includegraphics[width=0.8\linewidth]{fig.png}
\end{center}
   \caption{Qualitative results of FaceCraft on AFW.}
\label{fig:long}
\label{fig:onecol}
\end{figure}
 Examples of detection results are shown in Fig. 4. ~\cite{Alpher03} In our test results, non-frontal face bounding-box centred on the nose, which isconsistentwithourtrainingground-truthshowninFig.3. While in AFW ground-truth, nose is on the bounding-box edge ~\cite{Alpher02}. 
\bibliographystyle{ieee_fullname}
\bibliography{egbib}



%------------------------------------------------------------------------
\iffalse
\section{Final copy}

You must include your signed IEEE copyright release form when you submit
your finished paper. We MUST have this form before your paper can be
published in the proceedings.

Please direct any questions to the production editor in charge of these
proceedings at the IEEE Computer Society Press: Phone (714) 821-8380, or
Fax (714) 761-1784.

\fi



\end{document}
